\chapter{Finite Differenzen der stationären Gleichung}
Im folgendem Kapitel soll die stationäre Verteilung der Ladungsträger bei kontinuierlicher Bestrahlung modelliert werden. 
Dadurch kann die zeitliche Abhängigkeit vernachlässige werden ($\frac{\partial u}{\partial t}=0$)
 
Die allgemeine DGL ist gegeben durch:
\begin{equation}
	\frac{\partial u}{\partial t}= D\cdot\frac{\partial ^2 u }{\partial z^2}-(k1+k2\cdot N_D)\cdot u -k2u^2 +s(t,z)
\end{equation}

Mit $\frac{\partial u}{\partial t}=0$ folgt die stationäre Gleichung:

\begin{equation}\label{eq:stationDGL}
	D\cdot \frac{du}{dt} -\left( k_1 +k_2 N_D\right)\cdot u-k_2\cdot u^2=-s(z), \quad 0 <z<d
\end{equation}

mit den Randbedingungen:

\begin{equation}
	D\cdot \frac{\partial u}{\partial z}(0)=S_Lu(0),\quad D\frac{\partial u}{\partial z}(d)=-S_Ru(d)
\end{equation}
\section{Lineare stationäre Gleichung}
Im Folgendem Kapitel soll nur der in u lineare Anteil der stationären, zeitunabhängigen Gleichung (Eq. \ref{eq:stationDGL}) ohne den quadratischen Term $-k_2u^2$ behandelt werden\cite{Prof.Dr.AndreasZeiser.April2021}.

1. . Erarbeiten Sie sich Abschnitt 8.8 aus [1] und beschreiben Sie Ihr Vorgehen für die Anwendung der Methode auf Gleichung (6).

Mit der Verwendung dieser Methode auf die Gleichung 6 lässt sich ein Matrix berechnen, womit man die zeitabhängige Stelle der Leitungsträgerdichte u mathematisch beschreiben kann.dadurch kann ein Modell aus der Diskretisierung der Methode erstellt, das die lange dieser Ladungsträgerdichte an einer Zeit beschreibt.

2. Leiten Sie analog zu Gleichung (8.128) die Gleichungen für die gesuchten Werte  u0, u1, . . . , uN an den Punkten z0, . . . , zN her. Die Gleichungen enthalten $u_1 und u_\mathrm{N+1}$, die im nächsten Schritt eliminiert werden. Verwenden Sie dabei die Abkürzung $ si = s(zi)$.


\begin{equation}
	D\cdot \frac{du}{dt} -\left( k_1 +k_2 N_D\right)\cdot u-k_2\cdot u^2=-s(z), \quad 0 <z<d
\end{equation}\label{eq:stationDGL}
\begin{align*}
	k&=k_1+k_2*N_D\\
	s_i&=s(z_i)\\
	u(i)&\approx u_i
\end{align*}


\begin{equation}
	\frac{\partial u}{\partial z} = (u_i+1 - u_i)/(2\cdot h)
\end{equation}

\begin{equation}
	\frac{\partial ^2 u }{\partial z^2} = (u_i+1 - 2\cdot u_i + u_i-1)/(h^2) \quad i=1,2,\dots,N
\end{equation}

\begin{equation}
	D\cdot ((u_i+1 - 2\cdot u_i + u_i-1)/(h^2)) -k\cdot u = -s_i
\end{equation}

\begin{equation}
	(D/h^2)\cdot u_i-1 - (2D/h^2+k)\cdot u_i + (D/h^2)\cdot u_i+1 = -s_i
\end{equation}

