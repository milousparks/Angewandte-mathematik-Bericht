\chapter{Finite Differenzen der stationären Gleichung}
Im folgendem Kapitel soll die stationäre Verteilung der Ladungsträger bei kontinuierlicher Bestrahlung modelliert werden. 
Dadurch kann die zeitliche Abhängigkeit vernachlässige werden ($\frac{\partial u}{\partial t}=0$)
 
Die allgemeine DGL ist gegeben durch:
\begin{equation}
	\frac{\partial u}{\partial t}= D\cdot\frac{\partial ^2 u }{\partial z^2}-(k1+k2\cdot N_D)\cdot u -k2u^2 +s(t,z)
\end{equation}

Mit $\frac{\partial u}{\partial t}=0$ folgt die stationäre Gleichung:

\begin{equation}\label{eq:stationDGL}
	D\cdot \frac{du}{dt} -\left( k_1 +k_2 N_D\right)\cdot u-k_2\cdot u^2=-s(z), \quad 0 <z<d
\end{equation}

mit den Randbedingungen:

\begin{equation}
	D\cdot \frac{\partial u}{\partial z}(0)=S_Lu(0),\quad D\frac{\partial u}{\partial z}(d)=-S_Ru(d)
\end{equation}
\section{Lineare stationäre Gleichung}
Im Folgendem Kapitel soll nur der in u lineare Anteil der stationären, zeitunabhängigen Gleichung (Eq. \ref{eq:stationDGL}) ohne den quadratischen Term $-k_2u^2$ behandelt werden\cite{Prof.Dr.AndreasZeiser.April2021}.

1. . Erarbeiten Sie sich Abschnitt 8.8 aus [1] und beschreiben Sie Ihr Vorgehen für die Anwendung der Methode auf Gleichung (6).

Mit der Verwendung dieser Methode auf die Gleichung 6 lässt sich ein Matrix berechnen, womit man die zeitabhängige Stelle der Leitungsträgerdichte u mathematisch beschreiben kann.dadurch kann ein Modell aus der Diskretisierung der Methode erstellt, das die lange dieser Ladungsträgerdichte an einer Zeit beschreibt.

2. Leiten Sie analog zu Gleichung (8.128) die Gleichungen für die gesuchten Werte  u0, u1, . . . , uN an den Punkten z0, . . . , zN her. Die Gleichungen enthalten $u_1 und u_\mathrm{N+1}$, die im nächsten Schritt eliminiert werden. Verwenden Sie dabei die Abkürzung $ si = s(zi)$.


\begin{equation}
	D\cdot \frac{du}{dt} -\left( k_1 +k_2 N_D\right)\cdot u-k_2\cdot u^2=-s(z), \quad 0 <z<d
\end{equation}\label{eq:stationDGL}
\begin{align*}
	k&=k_1+k_2*N_D\\
	s_i&=s(z_i)\\
	u(i)&\approx u_i
\end{align*}

\begin{equation}
	\frac{\partial u}{\partial z} = (u_i+1 - u_i)/(2\cdot h)
\end{equation}

\begin{equation}
	\frac{\partial ^2 u }{\partial z^2} = (u_i+1 - 2\cdot u_i + u_i-1)/(h^2) \quad i=1,2,\dots,N
\end{equation}

\begin{equation}
	D\cdot ((u_i+1 - 2\cdot u_i + u_i-1)/(h^2)) -k\cdot u = -s_i
\end{equation}

\begin{equation}
	(D/h^2)\cdot u_i-1 - (2D/h^2+k)\cdot u_i + (D/h^2)\cdot u_i+1 = -s_i
\end{equation}

3. Approximieren Sie die ersten Ableitungen an den Randbedingungen:

\begin{equation}
	D\cdot \frac{\partial u}{\partial z}(0)=S_Lu(0),\quad D\frac{\partial u}{\partial z}(d)=-S_Ru(d)
\end{equation}
durch:

\begin{equation}
	u' = \frac{u_1-u_-1}{2*h}, \quad u'(d) = \frac{u_N+1-u_N-1}{2*h}
\end{equation}

und lösen Sie die Gleichungen nach $u_1$ bzw. $u_N+1$ auf. Setzen Sie diese Ausdrücke in die Gleichungen für die Knoten $z_0$ und $z_N$ der letzten Teilaufgabe ein.

Erste Randbedingungen:

\begin{equation}
	D\cdot \frac{\partial u}{\partial z}(0)=S_Lu(0)
\end{equation}

\begin{equation}
	\frac{\partial u(0)}{\partial z}=S_Lu(0)/D	
\end{equation}

\begin{equation}
	S_Lu(0)/D = \frac{u_1-u_-1}{2*h}
\end{equation}

\begin{equation}
	u(0) = D\cdot \frac{u_1-u_-1}{2*h}/S_L
\end{equation}

\begin{equation}
	u(0) = \frac{D}{2*h*S_L} * u_1 - \frac{D}{2*h*S_L} * u_-1
\end{equation}

\begin{equation}
	u_-1 = u_1 - \frac{2*h*S_L}{D} * u_(0)
\end{equation}

Zweite Randbedingungen:

\begin{equation}
	D\frac{\partial u}{\partial z}(d)=-S_Ru(d)
\end{equation}

\begin{equation}
	\frac{\partial u}{\partial z}(d)=-S_Ru(d)/D
\end{equation}

\begin{equation}
	-S_Ru(d)/D = \frac{u_N+1-u_N-1}{2*h}
\end{equation}

\begin{equation}
	u(d) = D\cdot \frac{u_N+1-u_N-1}{-2*h*S_R}
\end{equation}

\begin{equation}
	u(d) = \frac{D}{-2*h*S_R} * u_N+1 - \frac{D}{-2*h*S_R} * u_N-1
\end{equation}

\begin{equation}
	u_N-1 = \frac{-2*h*S_R}{D} \cdot u_(d) + u_N-1
\end{equation}

\begin{equation}
	u_N+1 = \frac{-2\cdot h\cdot S_R}{D} \cdot u_(N) + u_N-1
\end{equation}


für $z_0$:

\begin{equation}
	z_0: \frac{u_1-2\cdot h\cdot S_R/D\cdot u_0-2\cdot u_0+u_1}{h^2} \cdot D-k\cdot u_(0) = -s_(0)
\end{equation}


für $z_N$:

\begin{equation}
	z_N: \frac{u_N-1-2\cdot h\cdot S_R/D\cdot u_N-2\cdot u_N+u_N-1}{h^2}\cdot  D-k\cdot u_(N) = -s_(N)
\end{equation}

für $z_0$:

\begin{equation}
	\frac{u_1-2\cdot h\cdot S_L/D\cdot u_0-2\cdot u_0+u_1}{h^2} \cdot D-k\cdot u_(0) = -s_(0)\\
	\frac{D\cdot u_1}{h^2}-\frac{2\cdot S_L\cdot u_0}{h}-\frac{-2\cdot u_0}{h}+\frac{D\cdot u_1}{h^2}-k\cdot u_(0) = -s_(0)\\
	2\cdot \frac{D\cdot u_1}{h^2}-\frac{2\cdot S_L\cdot u_0}{h}-\frac{-2\cdot u_0}{h}-k\cdot u_(0) = -s_(0)\\
	2\cdot \frac{D}{h^2}\cdot u_1-(\frac{2\cdot S_L}{h}+\frac{-2}{h}+k)\cdot u_(0) = -s_(0)
\end{equation}

für $z_i$:


\begin{equation}
	D\cdot \frac{u_i-1-2\cdot u_i+u_i+1}{h^2}-k\cdot u_(i) = -s_(i)\\
	\frac{D\cdot u_i-1}{h^2}-\frac{2\cdot D\cdot u_i}{h^2}+\frac{-2\cdot u_0}{h}-k\cdot u_(i) = -s_(i)\\
	\frac{D\cdot u_i-1}{h^2}+\frac{D\cdot u_i+1}{h^2}-(\frac{-2\cdot D}{h}+k)\cdot u_(i) = -s_(i)\\
\end{equation}


für $z_0$:

\begin{equation}
	\frac{u_N-1-2\cdot h\cdot S_R/D\cdot u_N-2\cdot u_N+u_N-1}{h^2} \cdot D-k\cdot u_(N) = -s_(N)\\
	\frac{D\cdot u_N-1}{h^2}-\frac{2\cdot S_R\cdot u_N}{h}-\frac{2\cdot u_N\cdot D}{h}+\frac{D\cdot u_N}{h^2}-k\cdot u_(N) = -s_(N)\\
	2\cdot \frac{D\cdot u_N-1}{h^2}-(\frac{2\cdot}{h^2} + \frac{2\cdot S_R}{h} + k)\cdot u_(N) = -s_(N)\\
\end{equation}



4. Stellen Sie das lineare Gleichungssystem für die Größen  analog zu Gleichung (8.133) in Atkinson Han Elementary Numerical Analysis 2004 AuszugDatei und folgende dar:
$ A\cdot u = b,\quad u = [u_i],\quad b = [b_i]$
Ordnen Sie die Gleichungen analog zu den Knotenpunkten.

für $ u_1$:

\begin{equation}
	\frac{D\cdot u_1-1}{h^2}-(\frac{2\cdot D}{h^2} + k)\cdot u_1  + \frac{D\cdot u_1+1}{h^2} = -s_1\\
	\frac{D\cdot u_0}{h^2}-(\frac{2\cdot D}{h^2} + k)\cdot u_1  + \frac{D\cdot u_2}{h^2} = -s_1\\
\end{equation}

für $ u_N-1$:

\begin{align*}
	\frac{D\cdot u_-1-1}{h^2}-(\frac{2\cdot D}{h^2} + k)\cdot u_N-1  + \frac{D\cdot u_N-1+1}{h^2} &= -s_N-1\\
	\frac{D\cdot u_N-2}{h^2} - (\frac{2\cdot D}{h^2} + k)\cdot u_N-1  + \frac{D\cdot u_N}{h^2} &
	= -s_N-1
\end{align*}

Mit Hilfe der Matrixform $ A\cdot u = b $ lässt sich dieses Gleichungssystem mit Hilfe von Matlab durch  gelöst werden. Dazu wird die Koeffizientenmatrix


$ \begin{bmatrix}
	-\frac{(2*S_L}{h + 2} & \frac{2*D}{h^2} & ... & ... & ... & ... \\
	\frac{D}{h^2} & -\frac{(2*D}{h^2 + k)} & \frac{D}{h^2} & ... & ... & ... \\
	& ... & ... & ... &  &  \\
	&  &  &  &  &  \\
	&  &  & \frac{D}{h^2} & -\frac{(2*D}{h^2 + k))} & \frac{D}{h^2} \\
	&  &  &  & \frac{2*D}{h^2} & -\frac{(2}{h^2 + 2*S_R} 
\end{bmatrix}  $

der unbekannten Vektor:


und die rechte Seite:


bestimmt.


\chapter{Implizite Einschrittverfahren}

Steife Systeme treten aber auch in der Modellierung von Reaktionen in der Chemie auf. Als Beispiel soll
eine chemische Reaktion dreier Stoffe A;B;C aus [3] dienen:

$ A \rightarrow B,quad B + B \rightarrow C+ B,quad B + C \rightarrow A + C. $

Die Dynamik der Konzentrationen der einzelnen Komponenten können durch Anfangswertproblem


\begin{align*}
	A: y'_1 = -0.04y_1 + 10^4y_2y_2\\
	B: y'_2 =  0.04y_1 - 10^4y_2y_3\cdot - 3\cdot 10^7y^2_2
	C: y'_3 =                              3\cdot 10^7y^2_2
\end{align*}

mit den Anfangswerten

$ y_1(0) = 1, y^2_2(0) = 0, y_3(0) = 0 $

modelliert werden. Anhand dieses Systems sollen im folgenden steife Differenzialgleichungen untersucht
und ein effizientes numerisches Verfahren entwickelt werden.

\section{Entwicklung}

Ein allgemeines System von Anfangswertproblemen für eine Funktion y kann als folgendes geschrieben werden:

$ y' = f(t,y) ,quad t \in [a,b],quad y(a) = y_a $


Der Wert von y wird näherungsweise an den Zeitpunkten berechnet:

$ y^(i) \approx y(t_i),quad i = 0,1,....;N $

Das implizite Euler-Verfahren
$ y^(i+1) = y(i) + hf(t_i+1,y^(i+1)] $

Die implizite Trapezregel:
$ y^(i+1) = y(i) + h/2[f(t_i, y^(i) + f(t_i+1,y^(i+1)))] $



Aufgabe 1.
Erarbeiten Sie sich Abschnitt 8.4 aus Atkinson Han Elementary Numerical Analysis 2004 AuszugDatei und fassen Sie in eigenen Worten zusammen, was Sie unter steifen Differenzialgleichung verstehen.
Unter einer starren Differentialgleichung versteht man eine Gleichung mit bestimmten numerischen Methoden, die besonders große Störungsfaktoren am Ausgang aufweisen, sofern die Schrittweite nicht allzu klein ausgewählt wird.



Aufgabe 2:
Setzen Sie in den Gleichungen des impliziten Euler-Verfahrens und der impliziten Trapezregel
$ y^(i+1) = y(i) + z $

und formulieren Sie jeweils die Gleichung für  als Nullstellenproblem:
$ F_euler(z) = 0,quad F_trapez(z) = 0 $

$ y^(i+1) = y(i) + hf(t_i+1,y^(i+1)] $

und die implizite Trapezregel:
$ y^(i+1) = y(i) + h/2[f(t_i, y^(i) + f(t_i+1,y^(i+1)))] $

wir setzen nun
$ y^(i+1) = y^(i) + z $

in:

$ y^(i) + z = y^(i) + hf(t_i+1,y^(i) + z) $

und in:

$ y^(i) + z = y^(i) + h/2 [f(t_i, y^(i) + f(t_i+1,y^(i) + z)] $

ein. Formulieren wir nun beides als Nullstellenproblem für  erhalten wir

$ F_euler(z) = z - hf(t_i+1, y^(i) + z) = 0 $

und 

$F_trapez(z) =  z - h/2 [f(t_i, y^(i) + f(t_i+1,y^(i) + z)] = 0$

Aufgabe 3.
Bestimmen Sie die Jacobi-Matrizen  und  von  und  in Abhängigkeit von , d.h. der Jacobi-Matrix von  bezüglich .

$ D\cdot F_euler(z) = D\cdot z - h\cdot D_(y)\cdot f(t_i+1, y^(i) + z) $

für das implizite Euler-Verfahren und

$ D\cdot F_euler(z) = h/2\cdot [D_(y)\cdot f(t_i+1, y^(i)) +  h\cdot D_(y)f(t_i+1, y^(i) + z)] - D\cdot z $

für das implizite Trapez-Verfahren,

wobei D, f =

$ \begin{bmatrix}
	\frac{\partial f_1 }{\partial y_1} & \frac{\partial f_1 }{\partial y_2} & ... & \frac{\partial f_1 }{\partial y_n} \\
	\frac{\partial f_2 }{\partial y_1} & \frac{\partial f_2 }{\partial y_2} & ... & \frac{\partial f_2 }{\partial y_n} \\
	... & ... & ... & ... \\
	\frac{\partial f_n }{\partial y_1} & \frac{\partial f_n }{\partial y_2} & ... & \frac{\partial f_n }{\partial y_n} 
\end{bmatrix}  $

und Dz =

$ \begin{bmatrix}
	1 & 0 & ... & 0 \\
	0 & 1 & ... & 0 \\
	... & ... & ... & ... \\
	0 & ... & 0 & 1 
\end{bmatrix}  $


\section{Anwendung}

1. Formulieren Sie Gleichung (9) als System der Form (11) mit Funktion fchem(t; y).

$y' = f(t,y) = $
$ \begin{bmatrix}
	-0.04y_1 + 10^4y_2y_3 \\
	0.04y_1-10^4y_2y_3\cdot 10^7y^2_2 \\
	3 \cdot 10^7y^2_2 
\end{bmatrix}  $

mit $ t\in [0,\sim], y(0) = $ \begin{bmatrix}
	1 \\
	0 \\
	0 
\end{bmatrix}  $$
