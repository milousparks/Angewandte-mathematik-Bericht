\chapter{Finite Differenzen der stationären Gleichung}
Im folgendem Kapitel soll die stationäre Verteilung der Ladungsträger bei kontinuierlicher Bestrahlung modelliert werden. 
Dadurch kann die zeitliche Abhängigkeit vernachlässige werden ($\frac{\partial u}{\partial t}=0$)
 
Die allgemeine DGL ist gegeben durch:
\begin{equation}
	\frac{\partial u}{\partial t}= D\cdot\frac{\partial ^2 u }{\partial z^2}-(k1+k2\cdot N_D)\cdot u -k2u^2 +s(t,z)
\end{equation}

Mit $\frac{\partial u}{\partial t}=0$ folgt die stationäre Gleichung:

\begin{equation}\label{eq:stationDGL}
	D\cdot \frac{du}{dt} -\left( k_1 +k_2 N_D\right)\cdot u-k_2\cdot u^2=-s(z), \quad 0 <z<d
\end{equation}

mit den Randbedingungen:

\begin{equation}
	D\cdot \frac{\partial u}{\partial z}(0)=S_Lu(0),\quad D\frac{\partial u}{\partial z}(d)=-S_Ru(d)
\end{equation}
\section{Lineare stationäre Gleichung}
Im folgenden Kapitel soll nur der in $u$ lineare Anteil der stationären, zeitunabhängigen Gleichung (Eq. \ref{eq:stationDGL}) ohne den quadratischen Term $-k_2u^2$ behandelt werden\cite{Prof.Dr.AndreasZeiser.April2021}.

\begin{equation}
	D\frac{\partial ^2u}{\partial z^2}-kz=-s(z), \qquad 0<z<d,
\end{equation}



