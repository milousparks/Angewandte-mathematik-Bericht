\chapter{Implizite Einschrittverfahren}
In diesem Kapitel werden Methoden zur Lösung von sogenannten steifen Anfangswertproblemen behandelt. 
Dazu soll Folgendes Chemisches Problem anhand einer DGL untersucht werden.

Die zu untersuchenden Reaktionsgleichungen sind:
\begin{equation}
	A\rightarrow B, \quad B+B \rightarrow C+B, \quad B+C\rightarrow A+C
\end{equation}

Die Dynamik der Konzentrationen der Komponenten werden durch folgende Anfangswertprobleme beschrieben:
\begin{align*}
	A&:y'_1&=-0.04 & y_1+10^4 & y_2y_3\\
	B&:y'_2&=0.04 & y_1-10^4 & y_2y_3-3\cdot 10^7y_2^2\\
	C&:y'_3&=\\
\end{align*}

\section{Entwicklung}
\section{Anwendung}



