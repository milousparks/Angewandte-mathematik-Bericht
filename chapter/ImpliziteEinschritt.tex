\chapter{Implizite Einschrittverfahren}
In diesem Kapitel werden Methoden zur Lösung von sogenannten steifen Anfangswertproblemen behandelt. 
Dazu soll Folgendes Chemisches Problem anhand einer DGL untersucht werden.

Die zu untersuchenden Reaktionsgleichungen sind:
\begin{equation}
	A\rightarrow B, \quad B+B \rightarrow C+B, \quad B+C\rightarrow A+C
\end{equation}

Die Dynamik der Konzentrationen der Komponenten werden durch folgende Anfangswertprobleme beschrieben:
\begin{align*}
	A:y'_1&= &{}-0.04\cdot y_1  &{}+ 10^4\cdot y_2y_3& & &\\
	B:y'_2&= & 0.04\cdot y_1  &{}- 10^4\cdot y_2y_3& &{}- 3\cdot 10^7y_2^2 &\\
	C:y'_3&= &                  &                  &  & 3\cdot 10^7y_2^2 &\\
\end{align*}
mit den Anfangswerten 


\section{Entwicklung}
\section{Anwendung}



